\documentclass[a4paper]{article}

\usepackage{onecolceurws}
\usepackage{amsmath}
\usepackage{mathtools}
\usepackage{amsfonts}
\usepackage{amssymb}
\usepackage{graphicx}
%\usepackage[table]{xcolor} % does this break @-columns in tables?
\usepackage{verbatim}
\usepackage{paralist}
\usepackage[hide]{ed} %for ednotes
\usepackage{listings,lstautogobble}
\lstset{
	basicstyle=\ttfamily,
	mathescape
}
\usepackage{paralist}
\usepackage{collectbox}
\usepackage{tabularx}
\usepackage{wrapfig}
\usepackage{lscape}
\usepackage{textcomp}
\usepackage{tikz}
\usepackage{tikz-cd}
\usetikzlibrary{cd}
\usepackage{pdfpages}
\usepackage{fancyvrb}
\usepackage{basics}
\addtolength{\intextsep}{-5mm}

\usepackage{geometry}
\geometry{
	a4paper,         % or letterpaper
	textwidth=14cm,  % llncs has 12.2cm
	textheight=24cm, % llncs has 19.3cm
	heightrounded,   % integer number of lines
	hratio=1:1,      % horizontally centered
	vratio=2:3,      % not vertically centered
}

%\newcommand{\mmt}{\texorpdfstring{{\normalfont\scshape{Mmt}}\xspace}{MMT\ }}

%\usepackage{basics} % Florian's macros
%\usepackage{local}
\usepackage[noadjust]{cite}
\renewcommand{\citedash}{--}    

\setcounter{tocdepth}{3}
\usepackage[bookmarks,linkcolor=red,citecolor=blue,urlcolor=gray,colorlinks,breaklinks,bookmarksopen,bookmarksnumbered]{hyperref}

\setlength{\hfuzz}{3pt}
\hbadness=10001 %make box warning less strict

\pagestyle{plain} % remove for final version


\begin{document}

\title{Leveraging Information Contained \\ in Theory Presentations}
\author{Yasmine Sharoda \\ sharodym@mcmaster.ca \\[0.3cm] Supervisors: Jacques 
	Carette and William Farmer}
\institution{Computing and Software, McMaster University, Canada} 
\maketitle
Building large libraries of mathematics is an active area of research in Mathematical Knowledge 
Management (MKM). A library structures information using some unit of knowledge. Whether that unit 
is called a theory, specification, module, locale or something else, we can abstractly see it as a way for 
encapsulating knowledge. 
We want to study the units of encapsulated information in formal systems, abstracting over the 
language details. Our abstraction consider units in which one can define sorts, typed symbols 
(optionally providing their definitions) and axioms stating the properties of the defined sorts and 
symbols. We use the term \textit{theory presentation} to refer to this abstract unit. The theory of this 
abstraction has been well-studied under the name universal algebra \cite{wechler1992universal}. Our 
goal is to identify what knowledge can be automatically generated from theory presentations. We want 
to provide a library of 
algorithms to generate this knowledge. This way, we provide tool support for library development and 
formalization tasks by removing some of the boilerplate associated with defining constructs that can 
be automatically generated. Not only does generating these new pieces save time and effort for the 
developer, but, more importantly, leverages the information contained in the theory graph by 
establishing new connections between the theories. A \textit{theory graph} is a collection of theories 
and 
theory morphisms. We want to be able to generate the new pieces of knowledge and the morphisms 
connecting them to the graph. We describe our contributions as 
\begin{itemize}
	\item Evaluating the status of current popular libraries of mathematics, in terms of modularity and 
	reusability.
	%	\item Studying the domain of theory presentations in different formal systems and analyzing 
	%their commonalities and variabilities. 
	\item Compiling a list of constructions that can be automatically generated from the syntax of a 
	theory presentation. 
	\item Building a library of generic algorithms for computing these constructions, given a theory 
	written in an abstract description language. At this point, we only consider 
	theory presentations in their most abstract format as a collection of sorts, function symbols and 
	axioms written in some formal language.  
	\item Implementing generators that can produce these constructions for a specific theory in a 
	specific language, taking into considerations all similarities and differences between different formal 
	systems.   
\end{itemize}

We now list some of the constructions that can be generated from theory presentations: \\ 
signatures, sub-algebras, product-algebras, projection, congruence, 
quotient algebra, record definition, homomorphism, homomorphism- 
equality, isomorphism, endomorphism, automorphism, composition-of-
morphisms, closed term language, open term language, staged-term-
language, structural induction principle, evaluation functions, 
rewrite rules, equivalence of terms, and parse trees. \\
Some of these constructions are inspired by universal algebra, others are more related to programming 
disciplines, like meta-programming. 

\paragraph{Approach}
We are interested in syntactically manipulating theory presentations, in their most abstract form. To 
achieve a high level of abstraction, we use \mmt \cite{kohlhase2010towards} as a data description 
language (DDL). 
\mmt is a foundation independent framework for managing mathematical knowledge. Knowledge is 
structured as a theory graph. 
Being foundation independent, it avoids the commitment to any specific logical foundation and allows 
the translation between the different ones. This makes it appealing to our research work as it enables 
us to abstract 
over language details, explore the minimal logic in which the different pieces of knowledge can be 
generated, and explore whether these pieces can be formalized in other logics. 

After generating the different pieces of knowledge, we want to be able to translate them into different 
formal systems. Considering systems with very different foundations, like Isabelle/HOL and Agda. 
The first 
is based on HOL, while the second is based on dependent type theory. They have different 
representations of theories, yet the information contained in them is the same. For example, a 
\verb|Monoid| 
theory consists of a sort, a point, a binary operation and three axioms. These pieces do not change 
based on the logic, but the way they are formalized does. We are interested in automating the 
generation of theories in different formal systems. For this, we aim to develop a feature model 
exploring the similarities and differences between the different systems. This feature model will be used 
to design generators from our abstract model to the different systems. This way, the 
information is generated once, and translated automatically to different representations. 

\paragraph{Evaluation}
We use the MathScheme library \cite{carette2011mathscheme} to test our implementation of 
combinators to generate information. The library contains over 1000 axiomatic theories, defined by 
using the combinators introduced in \cite{carette2012theory} and structured as a theory graph. To be 
able to use the library, we translate it into \mmt. The combinators used to build the library are 
interpreted as diagram level combinators \cite{rabediagram}. Theories in the library have different 
foundations. Some theories are based on simple 
logics, like equational logic, but others are based on more expressive\footnote{By expressivity we 
mean practical expressivity, in terms that information is expressed in a practical and convenient way} 
ones, like dependent type theory. 

\paragraph{} In conclusion, we aim at supporting the development of libraries by generating some 
related construction that can be made available by syntactic manipulation of theory presentations. As 
an outcome of our work, a user should be able to write something like 
\begin{lstlisting}
BooleanAlgebra := Theory {
  U : type;
  * : (U, U) -> U;
  + : (U, U) -> U;
  --> : (U, U) -> U;
  0 : U;
  1 : U;
  compl : U -> U;
  axiom unipotent_-->_1 : forall x : U. (x --> x) = (1);
  ....
}
generate (termLang, simplify)
\end{lstlisting}
and then ask the system to evaluate \verb|simplify ((x * y) + 1)| without any extra effort to define the 
term language or simplification rules. 

\bibliographystyle{splncs04}
\bibliography{proposal}

%\includepdf[pages=-]{yasmine.pdf}

\end{document}
